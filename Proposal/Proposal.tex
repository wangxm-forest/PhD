\documentclass[11pt,letter]{article}
\usepackage[top=1.00in, bottom=1.0in, left=1.1in, right=1.1in]{geometry}
\renewcommand{\baselinestretch}{1.5}
\usepackage{graphicx}
\usepackage{natbib}
\usepackage{amsmath}
\usepackage{hyperref}

\def\labelitemi{--}


\begin{document}
\bibliographystyle{besjournals}
\renewcommand{\refname}{\CHead{}}

\title{”A Fine Balance'': Exploring the Interplay of Reproductive Strategies and Growth Dynamics in Trees}
\author{Xiaomao Wang} 
\date{\today}
\maketitle

\setlength{\parindent}{0pt}
\setlength{\parskip}{3pt}

\section{Introduction}
Increasing temperature and changing precipitation patterns are influencing most of the plant species. These environmental changes can significantly affect their growth and reproductive activities. In particular, they might shifts how plants are going to invest their resources through growth-reproduction trade-offs, which highly affect future forest dynamics. One of these shifts is changes in variability across years in reproduction. Highly variability across years in reproduction is often associated with mast seeding, an important reproductive strategy.\par
Mast seeding is the phenomenon of synchronized and variable seed production within plant populations, it is closely related to ecosystem-level functions since seeds can be food resources for many mammals or host for some pathogens. It is widely obsered in wind-pollinated species, but also occurs in Poaceae as well as Diperocarpaceae. Even though there are several hypotheses explaining the potential mechanism of mast seeding (e.g., predator satiation, resource matching, etc.), the exact factors triggering this phenomenon in different species and ecosystems are not fully understood, but appear related to specific climatic conditions that allow trees to build up enough resources to mast. Thus, understanding what is related to masting and how climate change might affect masting will help us to know more about the forest regeneration dynamics globally. In this thesis, I will find what might be related to whether a species mast or not using reproductive traits, and link , and then trying to figure out how masting affect the growth reproduction trade-off.



\subsection{Thesis aims and research questions}
The aim of this thesis is to further understand how an important reproductive strategy—masting, affects the growth dynamics of tree species and influences forest dynamics in the context of climate change. In Chapter 2, I will investigate the relationship between different reproductive traits and the tendency of a species exhibiting masting behavior, aiming to identify critical traits for masting species. This analysis will help us determine which hypotheses most compellingly explain masting in different environments. In Chapter 3, I will analyze samples collected from Mt. Rainier to explore how species with different allocation strategies at different elevations apply varying masting strategies. In Chapter 4, I will examine whether there is a trade-off between growth and reproduction in these species,  and how masting influences the relationship between growth and reproduction, and what the primary triggers for masting are in this area. These results will enhance our understanding of how masting as a reproductive strategy affects the growth dynamics of these coniferous species.



\section{Explore which reproductive traits are relevant to masting}
Seed production is regulated by at least four developmental stages: floral transition, pollination, fertilization and seed maturation. 
\subsection{Research Objectives}
Which seed traits are related to masting, and what might cause masting in different species?
	\begin{enumerate}
	\item \textbf{Characterize Reproductive Traits:} Identify and categorize key reproductive traits in relation to masting behavior among different tree species.
	\item \textbf{Investigate Trait Associations:} Analyze the relationships between specific reproductive traits and masting event and frequency to determine which traits maybe advantages in masting.
	\item \textbf{Explore Ecological Contexts:} Investigate how environmental conditions and ecological interactions (seed predation), influence the reproductive traits linked to masting.
	\item \textbf{Develop Predictive Models:} Create predictive models that link seed traits with masting behavior to better understand potential responses to changing environmental conditions.
	\end{enumerate}
\subsection{Methods}
There are many hypotheses regarding mast seeding, including predator satiation, pollination efficiency, environmental prediction, weather cues, and the resource budget model. The traits we selected aim to fit different hypotheses of masting.

For the predator satiation hypothesis, relevant traits include seed size, seed nutrient content, dispersal potential, seed dormancy, and seed longevity. These traits are closely related to interactions with seed predators. Regarding pollination efficiency, the relevant traits are type of reproduction (monoecious or dioecious), flowering duration, and the types of pollinators. For the resource matching hypothesis, selected traits include leaf longevity and determinacy. Additionally, we will consider biome, continentality, and critical environmental cues to examine how these factors may influence whether a species will mast.

We will explore various databases to gather this trait data, including Silvics of North America, MASTREE+, the Kew Seed Dataset, and the European Atlas of Forest Tree Species.
\subsection{Significance} 
Studying the seed traits related to masting is significant for several reasons. First, it can help explain the potential mechanisms that drive this complex reproductive strategy, shedding light on why certain species mast while others do not. By examining traits such as seed size, nutrient content, dormancy, and longevity, researchers can gain insights into how these characteristics influence the success of seed production and dispersal during mast years, and how this is related to the seed predator's population dynamics.
Additionally, understanding these relationships can help us to predict how environmental factors, such as climate change, may impact masting behavior in the future. For example, knowing which seed traits are most relavant to masting behaviour in specific ecological contexts can help us anticipate how shifts in temperature, precipitation, and other environmental cues might change masting patterns. By answering the question of what causes masting through looking at seed traits, we can better understand the dynamics of forest ecosystems.

\section{Compare different species}
\subsection{Research Objectives}
How mast seeding events might differ for species with different resources allocation strategies.
	\begin{enumerate}
	\item \textbf{Compare Species Responses:} Evaluate the differences in masting and growth responses among various tree species across different elevations with various resources allocation strategies.
	\end{enumerate}
\subsection{Methods}
\begin{table}[htb]
	\centering
	\small
	\caption{Targeted species}
\begin{tabular}{|p{5cm}|p{5cm}|p{5cm}|}
\hline
 Tree species & Common name & Life history strategies\\ \hline
Abies amabilis & Pacific silver fir & Conservative \\ \hline
Pseudotsuga menziesii & Douglas fir & Flexible    \\\hline
Tsuga heterophylla & Western hemlock &     \\\hline
Tsuga mertensiana & Mountain hemlock &     \\\hline
Thuja plicata & Western redcedar & Conservative    \\\hline
Callitropsis nootkatensis & Yellow cedar & Conservative    \\\hline

\end{tabular}

\end{table}
\section{Understanding the relationship between masting and growth}
\subsection{Research Objectives}
Whether there is a trade-off between mast seeding events and growth.
	\begin{enumerate}
	\item \textbf{Assess Growth Patterns:} Analyze the historical annual growth of selected tree species in relation to their masting events to determine if there exist a trade-off.
	\item \textbf{Identify Environmental Influences:} Investigate how environmental factors (e.g., temperature, precipitation, etc.) affect the relationship between masting behavior and tree growth.	
	\end{enumerate}

\subsection{Methods}

To study the relationship between masting and growth, I have to ensure there is long-term seed data available. Our collaborator, Janneke Hille Ris Lambers, has been collecting stand-level seed data using seed traps starting from 2008 around Mt. Rainier, Washington, USA. To assess the historical growth of individual trees, I will collect cores from targeted species listed above in each stand with long-term seed data. Within each stand, there are at least 2 targeted species, and each species appears in at least 4 stands. I plan to collect tree cores from 25 individuals per species per stand, covering a size gradient. To make sure the trees we core were mature enough to reproduce at least 10 years ago, I will only collect from trees with DBH greater than 25cm. Additionally, I will prioritize individuals that are close to seed traps to better correlate with seed data. I will core trees at breast height using increment borers, taking two tree cores from the opposite sides of the tree while avoiding slopes. To account for at least 15 years of growth and for cross-dating purposes, I will collect long cores of 25 cm from larger trees with a DBH above 50 cm, while for smaller trees, I will core to the center if possible.

Collected tree cores will be sliced using microtome. I  will use a high-resolution camera to capture images of the tree cores and then measure the ring width with CooRecorder.

By combining averaged ring width data with seed data, we will be able to analyze the relationship between growth and reproduction for different species at various elevations.

\subsection{Significance}
Understanding the relationship between masting and growth is important for predicting how tree species will respond to changing environmental factors, particularly in the context of extending growing season, how trees might allocate additional carbon. By examining how these processes interact, we can gain insights into the adaptive strategies of different species and their potential resilience or vulnerability to future climate changes. This knowledge allows us to make more informed predictions about future forest dynamics, including shifts in species composition, regeneration, and ecosystem health. As climate conditions continue to fluctuate, understanding the  masting behavior and growth responses will be essential for effective forest management and conservation efforts, helping us to anticipate changes and make strategies that will support the sustainability and resilience of forest ecosystems.



\end{document}