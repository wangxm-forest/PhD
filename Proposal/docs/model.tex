
\documentclass[11pt,letter]{article}
\usepackage[top=1.00in, bottom=1.0in, left=1.1in, right=1.1in]{geometry}
\renewcommand{\baselinestretch}{1.5}
\usepackage{graphicx}
\usepackage{natbib}
\usepackage{amsmath}
\usepackage{hyperref}

\def\labelitemi{--}
\begin{document}
\textbf{Likelihood for Seed Production}\\
% emw2025May20 -- much better! But I also think missing some important parts that could be hard to work on from afar so I think best is that we discuss in June with Victor and Mike (see my request for a git issue below)? It's just a difficult model I think! That said, I think you should implement now suggestions about how to present the seed model (see below, 'please make these changes now and send back to me'). 
Growth data is measured on the individual tree-level, while seed data is collected via traps that capture multiple individual trees across a stand of trees. Thus I plan to model individual tree radial growth and stand-level seed production by introducing \(\mu_{j,sp,y}\), representing the mean radial growth for stand \(j\), species \(sp\), and year \(y\). % emw2025May20 -- Good! But, I think you need more here. You need to allow some variation in these ring widths attributable to something other than seed production. Otherwise you force the model to find variation with seed production too much. Thus, I think you need an additional set of equations for \mu_{j,sp,y} (and even \mu_{j,sp,y[i]}). It should probably include an effect of tree size and maybe also climate and/or tree neighborhood. Check out the OSPREE traitors paper (in the ospree repo, it's the paper that Deirdre is leading). 
% emw2025May20 -- Also, can you start a git issue for this task? I think it should include the following as a list of some sort of list:
% Expand growth model to include size (or similar)
% Consider what else must go in growth model -- are a few intercepts for stand and sp and year enough? Or do we need to model climate and other stuff explicitly?
% For seed model, do we need sp and stand separate or would one beta for sp x stand x year suffice?
% Okay that seeds depend on growth and not vice versa? 
% For seed model, given that each stand is at a unique elevation, is modeling both stand and elevation together okay?

For each individual tree \(i\) in stand \(j\), species \(sp\), and year \(y\), the observed radial growth is modeled as, with $\sigma$ being the variance across individuals:

\[
\text{Growth}_{j,sp,y[i]} \sim \text{Normal}(\mu_{j,sp,y}, \sigma^2)
\]

The seed count for a particular stand \(j\), species \(sp\), and year \(y\) follows a negative binomial distribution with mean \(\lambda_{j,sp,y}\) and dispersion term \(\phi\):

\[
\text{Seed}_{j,sp,y} \sim \text{NegBinomial}(\lambda_{j,sp,y}, \phi)
\]

The modeled seed count (\(\lambda_{j,sp,y}\)) for stand \(j\), species \(sp\), and year \(y\) is:
% emw2025May20 -- The above equation seems close, especially, once we update the growth equation. I think you need a growth model that estimates an average of tree growth at the stand level and an average of tree growth at the stand x species level -- they cannot be the same parameter as they are here. 

% emw2025May20 -- I would (please make these changes now and send back to me): (1) divide up how you explain the parameters into groups for your readers (I give an example for the first three, start this for all, even though we'll need to adjust), and (2) try to explain everything as simply and biologically as possible, then give the statistical terms (like slope and intercept, see example below). 
\[
\log(\lambda_{j, sp, y}) =\alpha_{0} + \alpha_{sp} + \alpha_{j} + (\beta_{1 sp} + \beta_{1 j}) \cdot \mu_{j,sp,y} + (\beta_{2 sp} + \beta_{2 j}) \cdot \mu_{j,sp,y-1} + \beta_{3 sp} \cdot \text{Elevation}_{j}
\]
\\
This approach allows variation in the seed count due to species and year---independent of tree growth---via \\
\(\alpha_{0}\): a baseline seed count (i.e., the grand mean across all species, stands and years not explained by any other components of the model);\\
\(\alpha_{sp}\): the effect of each species on this baseline (that is a species-specific intercept for each species $sp$, which is modeled as an offset from the baseline);\\ 
\(\alpha_{j}\): the effect of each stand on this baseline (that is a stand-specific intercept for each stand \(j\), which is modeled as an offset from the baseline);\\
\\
It also allows variation in seed count due to tree growth in both the current previous years:\\
\(\beta_{1 sp}\): the effect of current-year growth on seed production for species \(sp\) (that is a species-specific slope for the relationship between current-year growth and seed production);\\
\(\beta_{1 j}\): the effect of current-year growth on seed production for stand \(j\) (that is a stand-specific slope for the relationship between current-year growth and seed production);\\
\(\mu_{j,sp,y}\): the mean growth of species \(sp\) in stand\(j\) in year \(y\);\\
\(\beta_{2 sp}\): the effect of previous-year growth (lag effect) on seed production for species \(sp\) (that is a species-specific slope for the relationship between previous growth and seed production);\\
\(\beta_{2 j}\): the effect of previous-year growth (lag effect) on seed production for stand \(j\) (that is a stand-specific slope for the relationship between previous growth and seed production);\\
\(\beta_{3 sp}\): the effect of elevation on seed production for species \(sp\) (that is a species-specific slope for the relationship between elevation and seed production).\\
\\
\textbf{Priors for varying intercepts and slopes}\\
We specify normal priors for both species- and stand-level deviations in intercepts and slopes. These priors allow species and stands to vary around the overall mean relationships in a structured way:\\
\textbf{Species and stand-level intercepts}
\[
\alpha_{sp} \sim \text{Normal}(0, \sigma_{\alpha_sp})
\]
\[
\alpha_{j} \sim \text{Normal}(0, \sigma_{\alpha_j})
\]
\textbf{Species and stand-level slopes for growth effects}
\[
\beta_{1 sp} \sim \text{Normal}(0, \sigma_{\beta_1 sp})
\]
\[
\beta_{2 sp} \sim \text{Normal}(0, \sigma_{\beta_2 sp})
\]
\[
\beta_{1 j} \sim \text{Normal}(0, \sigma_{\beta_1 j})
\]
\[
\beta_{2 j} \sim \text{Normal}(0, \sigma_{\beta_2 j})
\]
\\
Where:\\ 
\(\sigma_{\alpha_sp}\): how much variation is expected from the baseline seed count across species (species level standard deviation);\\
\(\sigma_{\alpha_j}\): how much variation is expected from the baseline seed count across stand (stand level standard deviation);\\
\(\sigma_{\beta_1 sp}\) and \(\sigma_{\beta_2 sp}\): the extent of variation in these growth–seed production relationships across species (species level variation around the slope);\\
\(\sigma_{\beta_1 j}\) and \(\sigma_{\beta_2 j}\): the extent of variation in these growth–seed production relationships across stand (stand level variation around the slope).

\end{document}