\documentclass[11pt,letter]{article}
\usepackage[top=1.00in, bottom=1.0in, left=1.1in, right=1.1in]{geometry}
\renewcommand{\baselinestretch}{1.5}


\def\labelitemi{--}
\begin{document}

\title{Mao's first committee meeting outlines}
\author{Xiaomao Wang} 
\date{\today}
\maketitle

\setlength{\parindent}{0pt}
\setlength{\parskip}{3pt}

\section{Introduction of myself}
A brief introduction to my personal and educational background. %emw13Nov: Good! I would take a little while on this (3-5 minutes of a max 20 minute presentation) so people can get to know you. You had some good slides from your interview-visit that I suspect you could re-use. 
\section{Introduction of PhD project}
\subsection{General background}
Some generic information on mast seeding.
\begin{itemize}
	\item Different hypothesis
	\item Characteristics
	\item Ecological significances
	\end{itemize}
\subsection{Key questions and knowledge gap} %emw13Nov: Can you rephrase these as questions and connect to  PhD chapters?
 \begin{itemize}
	\item Explore which reproductive traits are relevant to masting
	\item Understanding the relationship between masting and growth
	\item Investigate how future climate change will affect masting
	%emw13Nov: Do you want to briefly introduce your study system of MORA for several of your chapters in a slide? I think that would be good as your committee will need help understanding the elevation gradient AND needs to know about the long-term seed trap data that you will have access to. 
 \end{itemize}
\subsection{Significance}
\subsection{Current progress} %emw13Nov: Definitely have a slide of MORA photos
 \begin{itemize}
 	\item Completed tree core collection from Mt. Rainier %emw13Nov: Hmm, I would not say completed, I would say what you have collected (XX cores across XX species and XX plots across XX elevation gradient) and that you hope it will be enough data but will see. 
 	\item Finished data scraping from one book
 	\item Drafted the research proposal
 \end{itemize}
\subsection{Plan for next year}
 \begin{itemize}
 	\item Finish all data scraping and sample process
 	\item Start constructing models and data analyzing
 \end{itemize}
\section{Timelines for milestones for PhD} %emw13Nov: I would make the proposal and quals ONE slide and leave time for discussion about HOW your committee members like to approach the proposal (how long to review a draft and when they have time etc.) as the timing is not just up to you given the help and training you need from your committee
\subsection{Research proposal}
 \begin{itemize}
	\item Approved by Jan, 2025 %emw13Nov: I would push this back at least to February 2025
	\item Send final draft by end of Nov, 2024
	\end{itemize}
\subsection{Comprehensive exam}
 \begin{itemize}
	\item Complete by Feb, 2025  %emw13Nov: I would push this back at least to March 2025 
	\item Scheduled by end of Dec, 2024 %emw13Nov: Good!
	\end{itemize}
\subsection{Mast-trait manuscript} %emw13Nov: the rest of this outline should be ONE slide -- for a GAANT chart or similar, that you can linger on and discuss if you have time, or just show the committee and move on if your talk is running long. 
 \begin{itemize}
	\item Submit by June, 2026
	\item Start manuscript writing from Nov, 2025
	\item Start data analysis from Mar, 2025
	\item Finish data scraping by Feb, 2025
	\end{itemize}
\subsection{Trade-off manuscript}
 \begin{itemize}
	\item Submit by July, 2027
	\item Start manuscript writing from Jan, 2027
	\item Finish data analysis by Dec, 2025
	\item Finish tree core processing by Apr, 2025
	\end{itemize}
\subsection{Defense}	
\begin{itemize}
	\item Complete defense by May, 2028
	\item Schedule defense by Mar, 2028
	\end{itemize}
\subsection{Dissertation}	
\begin{itemize}
	\item Submit final dissertation by Jun, 2028
	\item Send final draft out by Mar, 2028
	\item Finish first draft by Dec, 2027
	\item Start writing from Jun, 2026
	\end{itemize}
\section{Answer questions}
\end{document}