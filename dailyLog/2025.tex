\documentclass[12pt]{article}
\usepackage[a4paper, margin=1in]{geometry}
\usepackage{titlesec}
\usepackage{datetime2}
\usepackage{xcolor}
\usepackage{hyperref}
\usepackage{parskip}  % better paragraph spacing

\titleformat{\section}{\normalfont\Large\bfseries}{\thesection}{1em}{}
\titleformat{\subsection}{\normalfont\large\bfseries}{\thesubsection}{1em}{}

% Customize hyperlink colors
\hypersetup{
    colorlinks=true,
    linkcolor=blue,
    urlcolor=blue,
    citecolor=blue
}

\title{\textbf{2025 PhD Daily Log}}
\author{Mao}

\begin{document}

\maketitle
\tableofcontents
\newpage

% -------- Day Entry Example --------
\section*{September 16, 2025}
\addcontentsline{toc}{section}{September 16, 2025}

\subsection*{Tasks for the Day}
\begin{itemize}
    \item Finish the phylo tree for mastTrait
    \item Bayes meeting
    \item Figured out the DPI problem
\end{itemize}
\subsection*{Notes / Study Summary}
Figured out how to make a circular tree. You can do it in the fancy way by using ggtree package, but I didn't get it working for me sadly. I used the basic plot function, just make the type to be ``fan'', and you can get a circular tree!\\
Jenna presented at the Bayes meeting today on bumble bee foraging. Her key question is that she has data collected at different time which is closely related to flora resources quality etc.. The final suggestion she got is adding a weighed term account for the proportion of data collected in certain time.\\
Didn't really get too deep into the DPI problem, I will try to test on a few tmr see if it worth doing the hand measuring or just redo all the scanning.
\subsection*{Challenges / Questions}
\begin{itemize}
    \item Do I want to calculate the DPI, which might not be that accurate. Or re-scan everything...
\end{itemize}

\subsection*{Ideas / Next Steps}
\begin{itemize}
    \item I will do some calculation on DPI for now, let's see if it works...
\end{itemize}

\vspace{1em}
\hrule
\vspace{1em}

\section*{September 17, 2025}
\addcontentsline{toc}{section}{September 17, 2025}

\subsection*{Tasks for the Day}
\begin{itemize}
    \item Finish the phylo tree for mastTrait
    \item Figured out the DPI problem
\end{itemize}
\subsection*{Notes / Study Summary}
Made the tree with masting info for mastTrait and updated the tree for egret. Learned some new plotting techniques which I will add to my cheatsheet soon.\\
Ended up wfh today because I planned for lunch with Miah, I will continue with the DPI problem tmr.

\vspace{1em}
\hrule
\vspace{1em}

\section*{September 18, 2025}
\addcontentsline{toc}{section}{September 18, 2025}

\subsection*{Tasks for the Day}
\begin{itemize}
    \item Clean the flowering time data
    \item Finished the first chapter of Probability
    \item Prepare some flash card on cone reproduction
\end{itemize}
\subsection*{Notes / Study Summary}
I ended up realize that there are more data available on the USDA Fire Effects Information System website, including more masting information, soil and climate info. I decided that I would record the climate (moisture and temperature) and soil (dry/wet) to account for the different environments. I also read about the vegetation types in North America which I wonder if I can also use for my analysis. Anyway, I think the collecting and cleaning data might take me another week.\\
I finally decided to rescan all the tree cores just for a more confirmed DPI without introducing too much error.\\
\subsection*{Challenges / Questions}
\begin{itemize}
    \item How to classify these species into different environment categories? Now I have climate, soil (dry/wet), biome and elevation maybe?
\end{itemize}

\vspace{1em}
\hrule
\vspace{1em}
\section*{September 19, 2025}
\addcontentsline{toc}{section}{September 19, 2025}

\subsection*{Tasks for the Day}
\begin{itemize}
    \item Continue getting more data for mastTrait
    \item Catch up on climategrowth repo
\end{itemize}
\subsection*{Notes / Study Summary}
Got more data for mastTrait, decided to get data on whether they have an effective seedbank as well since that data is available on USDA. This can be used as an indicator of seed longevity maybe.\\
Didn't really get too much on the climategrowth repo, I think I am a bit confused.
\subsection*{Challenges / Questions}
\begin{itemize}
    \item I think I have too much to catch up on climategrowth repo.
\end{itemize}

\subsection*{Ideas / Next Steps}
\begin{itemize}
    \item Maybe try to read the issues and talk with Victor?
\end{itemize}

\vspace{1em}
\hrule
\vspace{1em}

\section*{September 22, 2025}
\addcontentsline{toc}{section}{September 22, 2025}

\subsection*{Tasks for the Day}
\begin{itemize}
    \item Continue getting more data for mastTrait
    \item Catch up on climategrowth repo
    \item Training on 3322
    \item 
\end{itemize}
\subsection*{Notes / Study Summary}
Got more data for mastTrait, found another new website providing some lifespan and reproduction age data: The Gymnosperm Database. Lizzie ask for generation time for some species on climategrowth project and gave me another paper to look at.\\
I finally got back on scanning again. To make sure we can always check the DPI later, we captured an image of the scaler at the end of the batch. However, it seemed that adding the scaler as a sample was a bad idea. I only need to capture single sample and it would appear in the folder.\\
Didn't have time to work on egret

\vspace{1em}
\hrule
\vspace{1em}

\section*{September 23, 2025}
\addcontentsline{toc}{section}{September 23, 2025}

\subsection*{Tasks for the Day}
\begin{itemize}
    \item Continue getting more data for mastTrait
    \item Refine egretTree, finish my tasks on USDA
\end{itemize}
\subsection*{Notes / Study Summary}
I prioritized the egret tasks for today, finalized on the color palette for my tree, and checked on the tables assigned to me. I also set up some tree cores for scanning.
\subsection*{Challenges / Questions}
\begin{itemize}
    \item The tree cores were scanned, maybe because it was upset with the 0,0 which I used to capture the scaler?
    \item USDA cleaning code seems to be a little bit messy especially with the fact that some columns' names got changed.
\end{itemize}

\vspace{1em}
\hrule
\vspace{1em}
\section*{September 24, 2025}
\addcontentsline{toc}{section}{September 24, 2025}

\subsection*{Tasks for the Day}
\begin{itemize}
    \item Meeting with Jenn for the reading list
    \item Egret meeting
    \item Continue with datascraping and cores
\end{itemize}
\subsection*{Notes / Study Summary}
Had a great meeting with Jenn today, she recommended me to read something on the life-history and trade-offs. I found the books she recommended available in the library.\\
Had the egret meeting today. We discussed about potential better management of the pipeline cleaning code. We should've had all the manual cleaning (using the which statement) in the very beginning in the cleaning code before we change any column names and make further cleaning using code. I think I can fix this.\\
The core scanning was successful, but my suspesion is that the light was too dark, I turned up the light and started another batch. Hopefully it will work better.
\subsection*{Challenges / Questions}
\begin{itemize}
    \item Cores not able to be scanned.
\end{itemize}

\subsection*{Ideas / Next Steps}
\begin{itemize}
    \item Changed the light setup.
\end{itemize}

\vspace{1em}
\hrule
\vspace{1em}
\section*{September 25, 2025}
\addcontentsline{toc}{section}{September 25, 2025}

\subsection*{Tasks for the Day}
\begin{itemize}
    \item Continue with datasraping and cores
    \item Reading for comps
\end{itemize}
\subsection*{Notes / Study Summary}
I did something really stupid...I accidentally deleted everything operating TINA, which I didn't realized and could not make TINA work again. Even though everything is backup-ed on GitHub, but I still feel frustrated . I think me personally don't know Linux command well enough, and it is actually quite dangerous for me to use command to operate TINA. I wanted to learn Linux command many years ago but never got a chance to start. I felt that my learning ability is getting worse and slower which makes me feel sad and helpless. Anyway, this is a daily log, not your journal. I already contacted Adam and he said he would come over to check on TINA next week when he is on campus.\\
Couldn't get the books Jenn recommended, I guess I'll have to actually go through those shelves and find the books myself. 
\subsection*{Challenges / Questions}
\begin{itemize}
    \item TINA IS NOT WORKING
\end{itemize}

\subsection*{Ideas / Next Steps}
\begin{itemize}
    \item ADAM PLEASE HELP STUPID ME
\end{itemize}

\vspace{1em}
\hrule
\vspace{1em}
\section*{September 26, 2025}
\addcontentsline{toc}{section}{September 26, 2025}

\subsection*{Tasks for the Day}
\begin{itemize}
    \item Getting books for comps
    \item More data
\end{itemize}
\subsection*{Notes / Study Summary}
It was a hectic day because I tried to fix TINA but still failed. It kept showing an error which I couldn't understand. I decided to ask Adam for help next week.\\
On a more cheerful note, I finally got the two books Jenn recommended from the library. The life history evolution book looks different than the one Jenn showed me, which I later realized was actually not the correct one!

\vspace{1em}
\hrule
\vspace{1em}

\section*{September 29, 2025}
\addcontentsline{toc}{section}{September 29, 2025}

\subsection*{Tasks for the Day}
\begin{itemize}
    \item Continue with trait data collecting
    \item Reading for comps
    \item Think about the model for the first chapter
\end{itemize}
\subsection*{Notes / Study Summary}
I was trying to trouble shooting TINA together with Adam today. It seems the other lab currently building TINA run into the same problem. Adam got TINA working eventually. Thank you Adam!\\
I met with Lizzie in the afternoon, and she suggested I prepare my talk and practice earlier. I should also schedule a practice comps with people who have had comps before, so I'm thinking about inviting Avery, Deirdre and maybe Eva to help me with this.\\
I did more on the trait data cleaning, it is really messy. Especially for the flowering traits, I have no idea how to fix this...
\subsection*{Challenges / Questions}
\begin{itemize}
    \item How to deal with the flowering traits? More specifically, the flowering time.
\end{itemize}

\subsection*{Ideas / Next Steps}
\begin{itemize}
    \item I don't actually see a big variation in the flowering time. For most of the temperate species, they seem flower during the spring, the difference is more linked to elevation and latitude.
\end{itemize}

\vspace{1em}
\hrule
\vspace{1em}

\section*{September 30, 2025}
\addcontentsline{toc}{section}{September 30, 2025}

\subsection*{Tasks for the Day}
\begin{itemize}
    \item Continue with trait data collecting
    \item Reading for comps
\end{itemize}
\subsection*{Notes / Study Summary}
OMG, I finally thought about TRY and hopefully I can get more useful data there! Request data is not that difficult at all. This is the power of open-access and open data.

\vspace{1em}
\hrule
\vspace{1em}

\section*{October 1, 2025}
\addcontentsline{toc}{section}{October 1, 2025}

\subsection*{Tasks for the Day}
\begin{itemize}
    \item Continue with trait data cleaning
    \item Prepare the slides for the talk
\end{itemize}
\subsection*{Notes / Study Summary}
I added the TRY trait data in the scraped data, and finished some data cleaning. We decided that conifers are so phylogenically different from other species, so we decided to look at them separately for now and maybe analyze them separately later as well. The monoecious is still the strongest signal for masting.\\
I also finished the draft for the presentation slides. It definitely still need some edits, but I have an outline for my talk now, which I could possibly use for my comps talk as well.

\vspace{1em}
\hrule
\vspace{1em}

\section*{October 2, 2025}
\addcontentsline{toc}{section}{October 2, 2025}

\subsection*{Tasks for the Day}
\begin{itemize}
    \item Talk practice
    \item Study for comps
\end{itemize}
\subsection*{Notes / Study Summary}
Had the talk practice with Victor and Lizzie today. They gave me lots of useful feedback, and there's one thing I think it's my current biggest short-come when doing presentations. It's that I am not giving the talk base on the slides. My slides are not very engaging and it has too much information. This is a long time problem for me since my undergrad. I think I'll try to think about how to tell an engaging story from my slides and make it more intuitive.

\vspace{1em}
\hrule
\vspace{1em}

\section*{October 3, 2025}
\addcontentsline{toc}{section}{October 3, 2025}

\subsection*{Tasks for the Day}
\begin{itemize}
    \item Finish the phylogeny tree of mastTrait
    \item Make some edits on the slides
\end{itemize}
\subsection*{Notes / Study Summary}
Added dormancy status, monoecious/dioeciou, and seed size to the phylogeny tree.\\
Made some small edits on the slides, more specifically, added some illustrations for masting.

\vspace{1em}
\hrule
\vspace{1em}
\section*{October 5, 2025}
\addcontentsline{toc}{section}{October 5, 2025}

\subsection*{Tasks for the Day}
\begin{itemize}
    \item Finish the slides for Bayesian meeting
\end{itemize}
\subsection*{Notes / Study Summary}
Found a cute video on Youtube where I stole some illustrations for masting. Almost done with the Bayesian meeting slides. I definitely need to think harder on the model itself.

\vspace{1em}
\hrule
\vspace{1em}

\section*{October 6, 2025}
\addcontentsline{toc}{section}{October 6, 2025}

\subsection*{Tasks for the Day}
\begin{itemize}
    \item Meet with Lizzie and get some feedback on my slides
\end{itemize}
\subsection*{Notes / Study Summary}
Lizzie asked me to try hard to stop getting more data, but concentrate on analyzing the data. There are still something I can potentially get out from a not very exciting project, like a new perspective, or drawing people's attention on missing data. I think I feel better after taking to Lizzie, but I do need to think more clearly on what's my perspectives will be, using current data I have.

\vspace{1em}
\hrule
\vspace{1em}

\section*{October 7, 2025}
\addcontentsline{toc}{section}{October 7, 2025}

\subsection*{Tasks for the Day}
\begin{itemize}
    \item Bayesian meeting and get some help on idea for modeling
    \item Finish some grading for the course
\end{itemize}
\subsection*{Notes / Study Summary}
Got lots of useful feedback from people during the Bayesian meeting. Have to think more on the correlation matrix for species and consider trait-transition matrix analysis.\\
Almost done with the grading.

\vspace{1em}
\hrule
\vspace{1em}

\section*{October 8, 2025}
\addcontentsline{toc}{section}{October 8, 2025}

\subsection*{Tasks for the Day}
\begin{itemize}
    \item Finalize grading
    \item TA training
    \item egret pop-ups
    \item start working on model for mastTrait
\end{itemize}
\subsection*{Notes / Study Summary}
Finished the grading today, with some tree ID I was unsure, so I will ask for help later.\\
The TA training was on how to moderate time and attempts on canvas, potentially helpful for the upcoming quiz.\\
Talked with Victor on provenance model, and we agreed that this is going to be a relatively easy task. I also sent Deirdre and Dan a separate email to schedule the meeting for our pop-up.\\
Submitted some updates on mastTrait, reread the paper on trait and masting on Nature, thought their model might be helpful to solve my problem.


\vspace{1em}
\hrule
\vspace{1em}

\section*{October 9, 2025}
\addcontentsline{toc}{section}{October 9, 2025}

\subsection*{Tasks for the Day}
\begin{itemize}
    \item Study for comps
    \item Experiment
\end{itemize}
\subsection*{Notes / Study Summary}
Went in to help with covering the growth trays in the growth chamber, but the plastic wrap didn't really work very well, so I ordered some ziplop bags instead, which means we have to postpone the germination experiment!\\
Printed some reading from Amy, she seemed to be very strongly-opinionated on the space-to-time. I should be prepared.
\vspace{1em}
\hrule
\vspace{1em}

\section*{October 10, 2025}
\addcontentsline{toc}{section}{October 10, 2025}

\subsection*{Tasks for the Day}
\begin{itemize}
    \item Study for comps
    \item Finish the slides for ETH talk
\end{itemize}
\subsection*{Notes / Study Summary}
Refine the slides for ETH talk, still need to work on the reproductive cycles for the targeted species. I think I've worked out a logical flow here, which makes me happy.\\
Started some readings Amy asked me to read, luckily I have two readings that are similar to the readings we read for CONS310.
\vspace{1em}
\hrule
\vspace{1em}
\end{document}
