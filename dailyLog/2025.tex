\documentclass[12pt]{article}
\usepackage[a4paper, margin=1in]{geometry}
\usepackage{titlesec}
\usepackage{datetime2}
\usepackage{xcolor}
\usepackage{hyperref}
\usepackage{parskip}  % better paragraph spacing

\titleformat{\section}{\normalfont\Large\bfseries}{\thesection}{1em}{}
\titleformat{\subsection}{\normalfont\large\bfseries}{\thesubsection}{1em}{}

% Customize hyperlink colors
\hypersetup{
    colorlinks=true,
    linkcolor=blue,
    urlcolor=blue,
    citecolor=blue
}

\title{\textbf{2025 PhD Daily Log}}
\author{Mao}

\begin{document}

\maketitle
\tableofcontents
\newpage

% -------- Day Entry Example --------
\section*{September 16, 2025}
\addcontentsline{toc}{section}{September 16, 2025}

\subsection*{Tasks for the Day}
\begin{itemize}
    \item Finish the phylo tree for mastTrait
    \item Bayes meeting
    \item Figured out the DPI problem
\end{itemize}
\subsection*{Notes / Study Summary}
Figured out how to make a circular tree. You can do it in the fancy way by using ggtree package, but I didn't get it working for me sadly. I used the basic plot function, just make the type to be ``fan'', and you can get a circular tree!\\
Jenna presented at the Bayes meeting today on bumble bee foraging. Her key question is that she has data collected at different time which is closely related to flora resources quality etc.. The final suggestion she got is adding a weighed term account for the proportion of data collected in certain time.\\
Didn't really get too deep into the DPI problem, I will try to test on a few tmr see if it worth doing the hand measuring or just redo all the scanning.
\subsection*{Challenges / Questions}
\begin{itemize}
    \item Do I want to calculate the DPI, which might not be that accurate. Or re-scan everything...
\end{itemize}

\subsection*{Ideas / Next Steps}
\begin{itemize}
    \item I will do some calculation on DPI for now, let's see if it works...
\end{itemize}

\vspace{1em}
\hrule
\vspace{1em}

\section*{September 17, 2025}
\addcontentsline{toc}{section}{September 17, 2025}

\subsection*{Tasks for the Day}
\begin{itemize}
    \item Finish the phylo tree for mastTrait
    \item Figured out the DPI problem
\end{itemize}
\subsection*{Notes / Study Summary}
Made the tree with masting info for mastTrait and updated the tree for egret. Learned some new plotting techniques which I will add to my cheatsheet soon.\\
Ended up wfh today because I planned for lunch with Miah, I will continue with the DPI problem tmr.

\vspace{1em}
\hrule
\vspace{1em}

\section*{September 18, 2025}
\addcontentsline{toc}{section}{September 18, 2025}

\subsection*{Tasks for the Day}
\begin{itemize}
    \item Clean the flowering time data
    \item Finished the first chapter of Probability
    \item Prepare some flash card on cone reproduction
\end{itemize}
\subsection*{Notes / Study Summary}
I ended up realize that there are more data available on the USDA Fire Effects Information System website, including more masting information, soil and climate info. I decided that I would record the climate (moisture and temperature) and soil (dry/wet) to account for the different environments. I also read about the vegetation types in North America which I wonder if I can also use for my analysis. Anyway, I think the collecting and cleaning data might take me another week.\\
I finally decided to rescan all the tree cores just for a more confirmed DPI without introducing too much error.\\
\subsection*{Challenges / Questions}
\begin{itemize}
    \item How to classify these species into different environment categories? Now I have climate, soil (dry/wet), biome and elevation maybe?
\end{itemize}

\vspace{1em}
\hrule
\vspace{1em}
\section*{September 19, 2025}
\addcontentsline{toc}{section}{September 19, 2025}

\subsection*{Tasks for the Day}
\begin{itemize}
    \item Continue getting more data for mastTrait
    \item Catch up on climategrowth repo
\end{itemize}
\subsection*{Notes / Study Summary}
Got more data for mastTrait, decided to get data on whether they have an effective seedbank as well since that data is available on USDA. This can be used as an indicator of seed longevity maybe.\\
Didn't really get too much on the climategrowth repo, I think I am a bit confused.
\subsection*{Challenges / Questions}
\begin{itemize}
    \item I think I have too much to catch up on climategrowth repo.
\end{itemize}

\subsection*{Ideas / Next Steps}
\begin{itemize}
    \item Maybe try to read the issues and talk with Victor?
\end{itemize}

\vspace{1em}
\hrule
\vspace{1em}

\section*{September 22, 2025}
\addcontentsline{toc}{section}{September 22, 2025}

\subsection*{Tasks for the Day}
\begin{itemize}
    \item Continue getting more data for mastTrait
    \item Catch up on climategrowth repo
    \item Training on 3322
    \item 
\end{itemize}
\subsection*{Notes / Study Summary}
Got more data for mastTrait, found another new website providing some lifespan and reproduction age data: The Gymnosperm Database. Lizzie ask for generation time for some species on climategrowth project and gave me another paper to look at.\\
I finally got back on scanning again. To make sure we can always check the DPI later, we captured an image of the scaler at the end of the batch. However, it seemed that adding the scaler as a sample was a bad idea. I only need to capture single sample and it would appear in the folder.\\
Didn't have time to work on egret

\vspace{1em}
\hrule
\vspace{1em}

\section*{September 23, 2025}
\addcontentsline{toc}{section}{September 23, 2025}

\subsection*{Tasks for the Day}
\begin{itemize}
    \item Continue getting more data for mastTrait
    \item Refine egretTree, finish my tasks on USDA
\end{itemize}
\subsection*{Notes / Study Summary}
I prioritized the egret tasks for today, finalized on the color palette for my tree, and checked on the tables assigned to me. I also set up some tree cores for scanning.
\subsection*{Challenges / Questions}
\begin{itemize}
    \item The tree cores were scanned, maybe because it was upset with the 0,0 which I used to capture the scaler?
    \item USDA cleaning code seems to be a little bit messy especially with the fact that some columns' names got changed.
\end{itemize}

\vspace{1em}
\hrule
\vspace{1em}
\section*{September 24, 2025}
\addcontentsline{toc}{section}{September 19, 2025}

\subsection*{Tasks for the Day}
\begin{itemize}
    \item Meeting with Jenn for the reading list
    \item Egret meeting
    \item Continue with datascraping and cores
\end{itemize}
\subsection*{Notes / Study Summary}

\subsection*{Challenges / Questions}
\begin{itemize}
    \item 
\end{itemize}

\subsection*{Ideas / Next Steps}
\begin{itemize}
    \item 
\end{itemize}

\vspace{1em}
\hrule
\vspace{1em}

\end{document}
