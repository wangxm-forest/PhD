\documentclass[11pt,letter]{article}
\usepackage[top=1.00in, bottom=1.0in, left=1.1in, right=1.1in]{geometry}
\renewcommand{\baselinestretch}{1.5}
\usepackage{graphicx}
\usepackage{natbib}
\usepackage{amsmath}
\usepackage{hyperref}

\def\labelitemi{--}
\begin{document}

\title{`A Fine Balance': Exploring the Interplay of Reproductive Strategies and Growth Dynamics in Trees} %emw13Nov: Please fix quotes throughout -- we have open (``) and close (") quotes, which can be double, as shown, or single
\author{Xiaomao Wang} 
\date{\today}
\maketitle

\setlength{\parindent}{0pt}
\setlength{\parskip}{3pt}

\section{Introduction} %emw13Nov: to tense of 'to go' as future is colloquial and should be used sparingly in scientific writing
%emw13Nov: You need to introduce growth reproductive trade-offs BEFORE you mention them ... so stepping back to how plants need to make resource decisions to maximize fitness, then discuss growth, reproduction (and storage?) and then you can set up to introduce trade-offs. I think you will then have to introduce another paragraph to bridge to what you have on masting. Again, you may want to link to fitness....
Increasing temperature and changing precipitation patterns are influencing many plant species. These environmental changes can significantly affect their growth and reproductive activities. In particular, they might shifts how plants are invest their resources through growth-reproduction trade-offs, which highly affect future forest dynamics. One of these shifts is changes in variability across years in reproduction. Highly variability across years in reproduction is often associated with mast seeding, an important reproductive strategy \citep{pearse2016mechanisms}.\par
Mast seeding is the phenomenon of synchronized and variable seed production within plant populations, it is closely related to ecosystem-level functions since seeds can be food resources for many mammals or host for some pathogens  \citep{janzen1971seed, kelly1994evolutionary, davies2024seed}. It is widely obsered in wind-pollinated species, but also occurs in Poaceae as well as Diperocarpaceae  \citep{kelly2002mast}. Even though there are several hypotheses explaining the potential mechanism of mast seeding (e.g., predator satiation, resource matching, etc.)  \citep{koenig2021brief}, the exact factors triggering this phenomenon in different species and ecosystems are not fully understood, but appear related to specific climatic conditions that allow trees to build up enough resources to mast  \citep{pearse2016mechanisms}. Thus, understanding what is related to masting and how climate change might affect masting will help us to know more about the forest regeneration dynamics globally. 
%emw13Nov: Not sure you need the below.... 
% In this thesis, I will find what might be related to whether a species mast or not using reproductive traits, and link , and then trying to figure out how masting affect the growth reproduction trade-off.



\subsection{Thesis aims and research questions} %emw13Nov: rephrase below as list of research questions (with chapters mentioned). You seem to have them below so I would just move them up and mention them here and in each chapter. 
The aim of my thesis is to understand how an important reproductive strategy---masting---affects the growth dynamics of tree species and influences forest dynamics in the context of climate change. In Chapter 2, I will investigate the relationship between different reproductive traits and the tendency of species to have masting behavior, aiming to identify critical traits associated with masting. This analysis will help us determine which hypotheses most compellingly explain masting in various environments. In Chapter 3, I will examine whether there is a trade-off between growth and reproduction in these species and how masting influences this relationship. In Chapter 4, I will use climate data to explore how environmental factors affect the relationship between masting behavior and tree growth, as well as how future climate change may impact these dynamics. These results will enhance our understanding of how masting, as a reproductive strategy, influences the growth dynamics of coniferous species.


\section{Explore which reproductive traits are relevant to masting}
%emw13Nov: I think this would be better as more than one paragraph. In the first one you set up the hypotheses that could underlie mast seeding in a way that makes it clear there are links to traits. In the next paragraph you set up the (ideally contrasting) predictions about traits from the hypotheses. 
Understanding the reproductive traits associated with mast seeding is essential for understanding the mechanisms that drive this phenomenon. Seed reproduction is regulated through at least four critical stages: floral transition, pollination, fertilization and seed maturation \citep{satake2021studying}. Each stage presents unique opportunities that can influence the overall success of mast seeding. There are several hypotheses explaining the mechanism of mast seeding, such as predator satiation, resource matching and pollination coupling \citep{kelly1994evolutionary, kelly2002mast, crone2014resource}. These hypotheses offer insights into how mast seeding and reproductive success are closely related to each other in order to mitigate some ecological interactions. For example, predator satiation hypothesis suggests aht by overwhelming seed predators with a big seed crop, these plants increase their likelihood that these seeds will survive to germinate \citep{janzen1971seed, silvertown1980evolutionary}. In this case, seeds need to have certain characteristics for them to satisfy the seed predator as well as survive. Resouce matching suggests that plants adjust their reproductive output based on environmental cures, optimizing their seed production in favorable conditions. Pollination coupling links the timing of flowering and seed set, ensuring that pollination will maximize fertilization success \citep{bogdziewicz2017masting, bogdziewicz2020flowering}. 


\subsection{Research Objectives}
Which seed traits are related to masting, and what might cause masting in different species?
	\begin{enumerate}
	%emw13Nov: In this and ALL chapters, I like the steps, but I want them to be less general and vague -- what predictions will you test?
	\item \textbf{Characterize Reproductive Traits:} Identify and categorize key reproductive traits in relation to masting behavior among different tree species.
	\item \textbf{Investigate Trait Associations:} Analyze the relationships between specific reproductive traits and masting event and frequency to determine which traits maybe advantages in masting.
	\item \textbf{Explore Ecological Contexts:} Investigate how environmental conditions and ecological interactions (seed predation), influence the reproductive traits linked to masting. %emw13Nov: Can you explain how you will do this or what you mean more? 
	\item \textbf{Develop Predictive Models:} Create predictive models that link seed traits with masting behavior to better understand potential responses to changing environmental conditions. %emw13Nov: Predicting what? Species with no current data or ...? I am not sure you need this. 
	\end{enumerate}
\subsection{Methods}
%emw13Nov: The below is not methods ... 
There are many hypotheses regarding mast seeding, including predator satiation, pollination efficiency, environmental prediction, weather cues, and the resource budget model. The traits we selected aim to fit different hypotheses of masting.

%emw13Nov: This is somewhat methods.
For the predator satiation hypothesis, relevant traits include seed size, seed nutrient content, dispersal potential, seed dormancy, and seed longevity. These traits are closely related to interactions with seed predators \citep{janzen1971seed}. Regarding pollination efficiency, the relevant traits are type of reproduction (monoecious or dioecious), flowering duration, and the types of pollinators. For the resource matching hypothesis, selected traits include leaf longevity and determinacy. Additionally, we will consider biome, continentality, and critical environmental cues to examine how these factors may influence whether a species will mast.

I will explore various databases to gather this trait data, including Silvics of North America, MASTREE+ \citep{hacket2022mastree+}, the Kew Seed Dataset, and the European Atlas of Forest Tree Species.
\subsection{Significance} 
%emw13Nov: This is a very generic topic sentence, can you make one that is more specific to the actual paragraph?
Studying the seed traits related to masting is significant for several reasons. First, it can help explain the potential mechanisms that drive this complex reproductive strategy, shedding light on why certain species mast while others do not. By examining traits such as seed size, nutrient content, dormancy, and longevity, researchers can gain insights into how these characteristics influence the success of seed production and dispersal during mast years, and how this is related to the seed predator's population dynamics.
Additionally, understanding these relationships can help us to predict how environmental factors, such as climate change, may impact masting behavior in the future. For example, knowing which seed traits are most relavant to masting behaviour in specific ecological contexts can help us anticipate how shifts in temperature, precipitation, and other environmental cues might change masting patterns. By answering the question of what causes masting through looking at seed traits, we can better understand the dynamics of forest ecosystems.

\section{Understanding the relationship between masting and growth}
%emw13Nov: Good stuff below, but to needs to be re-ordered to set up for the reader more clearly what the GAP is and how your work will address it. There should also be more background literature on what we know so far. Check out the grephon repo on my github for a paper that has some good references. 
The growth-reproduction trade-off is a fundamental concept in plant ecology, suggesting that increased investment in one process often comes at the expense of the other \citep{stearns1998evolution}. For trees, the decision to allocate resources to a big crop of seed production during a masting year can impact their overall growth dynamics \citep{hacket2016tree}, especially for some coniferous trees that typically require two years to complete their reproductive cycle.

Moreover, mast seeding and growth responses can differ significantly among various tree species, particularly across different elevations where resource availability and environmental stresses may vary. Species with different allocation strategies will also respond differently in environment with different level of resources. %emw13Nov: please spell check before you ask me to review. 

This chapter aims to explore the relationship between mast seeding and growth of 6 coniferous tree species, emphasizing the trade-offs that shape the reproductive strategies and how these strategies may be influenced by varying ecological conditions.
\subsection{Research Objectives}
Whether there is a trade-off between mast seeding events and growth. %emw13Nov: Make this a question
	\begin{enumerate}
	\item \textbf{Assess Growth Patterns:} Analyze the historical annual growth of selected tree species in relation to their masting events to determine if there exist a trade-off.
	\item \textbf{Compare Species Responses:} Evaluate the differences in masting and growth responses among various tree species across different elevations with various resources allocation strategies.
	\end{enumerate}

\subsection{Methods}
\begin{table}[htb]
	\centering
	\small
	\caption{Targeted species}
\begin{tabular}{|p{5cm}|p{5cm}|p{5cm}|}
\hline
 Tree species & Common name & Life history strategies\\ \hline %emw13Nov: italicize latin binomials 
\textit{Abies amabilis} & Pacific silver fir & Conservative \\ \hline
\textit{Pseudotsuga menziesii} & Douglas fir & Flexible    \\\hline
\textit{Tsuga heterophylla} & Western hemlock &     \\\hline
\textit{Tsuga mertensiana} & Mountain hemlock &     \\\hline
\textit{Thuja plicata} & Western redcedar & Conservative    \\\hline
\textit{Callitropsis nootkatensis} & Yellow cedar & Conservative    \\\hline
\end{tabular}
\end{table}
To study the relationship between masting and growth, I have to ensure there is long-term seed data available. Our collaborator, Janneke Hille Ris Lambers, has been collecting stand-level seed data using seed traps starting from 2008 around Mt. Rainier, Washington, USA. To assess the historical growth of individual trees, I will collect cores from targeted species listed above in each stand with long-term seed data. Within each stand, there are at least 2 targeted species, and each species appears in at least 4 stands. I plan to collect tree cores from 25 individuals per species per stand, covering a size gradient. To make sure the trees we core were mature enough to reproduce at least 10 years ago, I will only collect from trees with DBH greater than 25cm. Additionally, I will prioritize individuals that are close to seed traps to better correlate with seed data. I will core trees at breast height using increment borers, taking two tree cores from the opposite sides of the tree while avoiding slopes. To account for at least 15 years of growth and for cross-dating purposes, I will collect long cores of 25 cm from larger trees with a DBH above 50 cm, while for smaller trees, I will core to the center if possible.

Collected tree cores will be sliced using microtome. I  will use a high-resolution camera to capture images of the tree cores and then measure the ring width with CooRecorder.

By combining averaged ring width data with seed data, we will be able to analyze the relationship between growth and reproduction for different species at various elevations.

\subsection{Significance}
Understanding the relationship between masting and growth is important for predicting how tree species will respond to changing environmental factors, particularly in the context of extending growing season, how trees might allocate additional carbon. By examining how these processes interact, we can gain insights into the adaptive strategies of different species and their potential resilience or vulnerability to future climate changes. This knowledge allows us to make more informed predictions about future forest dynamics, including shifts in species composition, regeneration, and ecosystem health. As climate conditions continue to fluctuate, understanding the  masting behavior and growth responses will be essential for effective forest management and conservation efforts, helping us to anticipate changes and make strategies that will support the sustainability and resilience of forest ecosystems.

\section{Investigate how changing climate will affect masting dynamics} %emw13Nov: It seems like this chapter is past and future. 
\subsection{Outline}
\textbf{Background}
\begin{enumerate}
\item Masting is influenced by various environmental factors such as temperature, precipitation.
\item To investigate how climate change may impact masting events in the future, we need to explore the relationship between environmental factors and mast seeding behavior.
\end{enumerate}	
\textbf{Research objectives}
\begin{enumerate}
\item Investigate how key environmental factors (e.g., temperature, precipitation) are related to masting events at Mt. Rainier.
\item Use the elevational gradient at Mt. Rainier as a proxy for changing climate conditions to compare how masting behavior varies across different elevations.
\item Develop predictive models of masting behavior across elevations to forecast future seed dynamics under changing climate conditions.
\end{enumerate}
\textbf{Methods}
\begin{enumerate}
\item Mt. Rainier, which has already experienced climate change, offers a unique opportunity to study its effects on masting behavior due to its  elevational gradient.
\item Climatic data + masting data
\item Model development
\end{enumerate}
\textbf{Significance}
\begin{enumerate}
\item Understand how environmental factors are related to masting, predict future seed dynamics in this area
\item Impact on Mammal Population Dynamics
\item Implications for Long-Term Forest Dynamics
\end{enumerate}
Mast seeding is influenced by various environmental factors. In the context of climate change, one significant aspect is the anticipated extension of the growing season, which may allow trees to allocate more carbon to growth and reproduction \citep{keenan2014net}. This shift could have profound implications for the intensity and frequency of masting events.

Mast seeding is characterized by four main features: interannual variation, synchrony, temporal autocorrelation, and frequency. These characteristics are all likely to be affected by climate change in different ways \citep{hacket2021climate}. This chapter aims to identify common patterns in masting that are influenced by environmental factors. By establishing a "climate change fingerprint" on masting, investigating the large variance in interannual masting variation, and exploring the relationship between spatial and temporal changes in masting patterns across elevational gradients, we can better understand how climate change will affect mast seeding on a larger scale.


\subsection{Research Objectives}
How does climate change affect masting behavior and growth at different elevations? %emw13Nov: I think your question is about climate then .. and you're using the elevational gradient as a proxy for climate change; you should set this up more clearly above and adjust here.  You also will be working in a system where the climate has ALREADY changed, so you should discuss that. 
	\begin{enumerate}
\item \textbf{Identify Environmental Influences:} Investigate how environmental factors (e.g., temperature, precipitation, etc.) affect the relationship between masting behavior and tree growth. Identify the potentially most critical climatic factor.	
\item \textbf{Develop Predictive Models:} Predict how trees growing at different elevations will respond to different future climate change scenarios.	%emw13Nov: This seems a good place for predictive models. But I think perhaps you need to separate out developing a model and forecasting (predicting from it) in this chapter. 
	\end{enumerate}
\subsection{Methods}
I will work to identify a "fingerprint" of climate change on masting patterns across species and elevations, using long-term seed production datasets from Mount Rainier alongside local climatic data. The focus will be on examining synchrony in masting events across elevations, as well as identifying common patterns and divergences in masting changes.
\subsection{Significance}
Changes in masting behavior can have cascading effects on forest dynamics, including seed predator interactions and the regeneration success of tree species. Since masting events can provide a large food source drives population dynamics of lots of mammals in the forest, changes of masting patterns will alter these ecological relationships, and further the overall structure of ecosystems. Fore example, if climate change causes masting events to be less synchronized across species or elevations, it will affect seed availability for animals that rely on them as food source, potentially impacting their survival and reprdocuation.
Masting will also affect the regeneration of tree species itself. Many tree species rely on mast seeding years to establish successful new seedlings in the face of competition and predation. If climate change affects the frequency or intensity of mast seeding events, the successful regeneration of some tree species might be adversly affected. Thus, understanding how these characteristics of masting might be influenced by climate change is essential for predicting the resilience of forest ecosystems.

\bibliographystyle{besjournals}
\bibliography{Proposal} 
\end{document}